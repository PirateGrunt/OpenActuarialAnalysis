\chapter{The data and analysis process}
This will form the structure for the rest of the book. I look at the process\footnote{It's not a cycle, by the way. Analytics is meant to have an end, which supports a decision or behavior change. You can repeat the analysis as data changes, or analyze something new, but it's not a cycle.} as follows:
\begin{itemize}
\item{Data structure}
\item{Data maintenence and audit}
\item{Reporting}
\item{Visualization}
\item{Analysis}
\item{Publication}
\end{itemize}

The first two elements relate primarily to data. For this, I use the metaphor of the construction of a building. The first step is to develop a rational blueprint. This is a plan for how the building will be constructed, with consideration for its purpose, its occupants, physical and legal limitations and the like.

We'll look at each of these individually.
\section{Data structure}
This is the most critical element for any enterprise and one over which actuaries have the least influence. This is because data structure in an insurance company is tightly linked with transactional data. 

\section{Data maintenence and audit}
I will treat this category broadly and include the subject of Extract, Transform and Load: ETL. Data maintenence is the practice of housing 