\chapter{The SampleData}
To facilitate the learning process and in keeping with this book's spirit as a practical guide, we will work with an actual set of data. This is not a trivial amount of information. It's meant to reflect the volume and scope with which an actuary could expect to work. It will be constructed automatically and randomly and you may reproduce it on your computer. To do so, you will need to have installled R and all of the required packages. Information on doing so may be found in INSERT REFERENCE HERE.

Most demographic information may be obtained from the US Census. These are huge datasets and a bit of patience is required to download them.

For each county, the number of insureds is distributed as a Poisson where N is equal to lambda times exposure. Exposures is equal to the county's population.