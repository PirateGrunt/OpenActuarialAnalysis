\chapter{The software you will use}
You do not need to install all of this software all at once. It will be useful to refer back to this section as needed. I will describe the most significant components. Other ancillary tools will follow at the end. Consistent with the focus on data first, we will talk about data storage and retrieval technologies before delving into analytics tools.
\section{PostgresSQL}
PostgreSQL\index{PostgreSQL} is among the preeminent open source relational databases. Its history stretches back to the 1990s and blah, blah, blah.
\section{MongoDB}
\section{Knime}
Knime sits at the boundary between data maintenence and analytics. It is useful as a structured ETL process, but also has some highly sophisticated exploratory analysis and more advanced data mining tools. It may be downloaded from SOME WEBSITE. Note that the tool itself doesn't install 
\section{R}
The explosion of popularity of R is blah, blah, blah. Here are the libraries that you will need:
\section{Python}
Python has been around since the early '90s, but only recently has it gotten much attention from the analytic community. The publication of the Python libraries numpy and pandas have had a great deal to do with that.
\section{TeX}
TeX is a markup language which will aid in writing professional summaries of your analytic work. This book was written in TeX.
\section{Others}
RapidMiner, Weka, GitHub